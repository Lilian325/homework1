\documentclass[a4paper,12pt]{article}

% --- 包引入 ---
\usepackage[utf8]{inputenc}
\usepackage{geometry}
\usepackage{listings}       % 用于代码展示
\usepackage{xcolor}         % 用于颜色
\usepackage{graphicx}       % 用于图片
\usepackage[hidelinks]{hyperref}       % 用于超链接
\usepackage{fancyhdr}       % 用于页眉页脚
\usepackage{ctex}           % 中文支持
\usepackage{enumitem}       % 用于自定义列表样式
\usepackage{float}          % 用于精确控制图片位置
\usepackage{caption}        % 用于图片标题格式

% --- 页面设置 ---
\geometry{left=2.5cm, right=2.5cm, top=2.5cm, bottom=2.5cm}

% --- COOL 语言高亮定义 ---
\lstdefinelanguage{Cool}{
	keywords={class, inherits, if, then, else, fi, while, loop, pool, let, in, case, of, esac, new, isvoid, not, true, false, self},
	keywordstyle=\color{blue}\bfseries,
	ndkeywords={Int, String, Bool, Object, IO, Main, BST, Node},
	ndkeywordstyle=\color{teal}\bfseries,
	identifierstyle=\color{black},
	sensitive=false,
	comment=[l]{--},
	morecomment=[n]{(*}{*)},
	commentstyle=\color{green!50!black}\itshape,
	stringstyle=\color{red},
	morestring=[b]",
}

\lstset{
	language=Cool,
	basicstyle=\small\ttfamily,
	numbers=left,
	numberstyle=\tiny\color{gray},
	frame=single,
	breaklines=true,
	captionpos=b,
	tabsize=4,
	showstringspaces=false
}

% --- 论文元数据 ---
\title{\textbf{COOL 语言二叉搜索树的设计与实现}}
\author{数据科学与大数据技术 \\ 学生姓名: [蒋雯丽] \\ 学号: [20238131061]}
\date{2025年9月30日}

\begin{document}
	
	\maketitle
	
	% --- GitHub 链接区域 ---
	\begin{center}
		\vspace{1cm}
		\Large \textbf{GitHub 仓库链接:} \\
		\vspace{0.5cm}
		\url{https://github.com/Lilian325/homework1} % 请替换为你的实际链接
		\vspace{1cm}
	\end{center}
	
	\hrule
	\vspace{1cm}
	
	\begin{abstract}
		本报告详细阐述了使用 COOL (Classroom Object Oriented Language) 语言实现二叉搜索树 (Binary Search Tree, BST) 的过程。该项目旨在通过构建复杂数据结构,深入理解 COOL 语言的面向对象特性、递归机制以及静态类型系统的约束。实验结果表明,所实现的代码成功完成了节点的插入、查找以及中序遍历功能,验证了算法的正确性和鲁棒性。
	\end{abstract}
	
	\newpage
	
	\section{设计与实现}
	
	\subsection{设计理念}
	本此实现遵循模块化的面向对象设计原则。由于 COOL 是一种强类型语言,我们将数据存储逻辑(Node)与数据结构接口(BST)分离。该设计大量依赖\textbf{递归}来处理树形结构,这与 COOL 语言对递归调用的良好支持相契合。
	
	\subsection{类结构设计}
	本实现主要包含三个核心类:
	
	\subsubsection{1. \texttt{Node} 类 (节点)}
	这是树的基本构建单元,负责存储数据和维持树的拓扑结构。
	\begin{itemize}[label={}]
		\item \textbf{继承关系}: 继承自 \texttt{IO} 类(便于调试输出)。
		\item \textbf{属性}:
		\begin{itemize}[label={}]
			\item \texttt{val : Int} - 存储整型数据。
			\item \texttt{left : Node} - 指向左子节点的引用。
			\item \texttt{right : Node} - 指向右子节点的引用。
		\end{itemize}
		\item \textbf{核心方法}:
		\begin{itemize}[label={}]
			\item \texttt{init(v : Int) : Node} - 初始化节点值。
			\item \texttt{insert(new\_val : Int) : Object} - 递归插入新值。
			\item \texttt{search(target : Int) : Bool} - 递归查找目标值。
			\item \texttt{print\_in\_order() : Object} - 递归执行左-根-右遍历打印。
		\end{itemize}
	\end{itemize}
	
	\subsubsection{2. \texttt{BST} 类 (树包装器)}
	该类作为对外的接口封装,管理树的根节点,并处理空树等边界情况。
	\begin{itemize}[label={}]
		\item \textbf{属性}: \texttt{root : Node} - 指向树的根节点(初始为 void)。
		\item \textbf{方法}: 将 \texttt{insert}, \texttt{search}, \texttt{print} 等操作委托给根节点处理,在此之前先进行非空检查。
	\end{itemize}
	
	\subsubsection{3. \texttt{Main} 类}
	程序的入口点,负责实例化 BST 对象并运行预设的测试用例。
	
	\subsection{核心算法实现}
	
	核心逻辑位于 \texttt{Node.insert} 方法中。由于 COOL 不支持像 C 语言那样的显式指针操作,我们通过修改对象的 \texttt{left} 或 \texttt{right} 属性来构建树。
	
	\begin{lstlisting}[caption={COOL 中的递归插入逻辑}, label={lst:insert}]
		insert(new_val : Int) : Object {
			if new_val < val then
			-- 如果左子节点为空(void),则在此处创建新节点
			if (isvoid left) then
			left <- (new Node).init(new_val)
			else
			-- 否则,向左子树递归
			left.insert(new_val)
			fi
			else if val < new_val then
			-- 右子树的对称逻辑
			if (isvoid right) then
			right <- (new Node).init(new_val)
			else
			right.insert(new_val)
			fi
			else
			self -- 忽略重复值
			fi fi
		};
	\end{lstlisting}
	
	\subsection{遇到的挑战与解决方案}
	
	\textbf{挑战:处理 \texttt{void} 引用} \\
	在 COOL 中,未初始化的对象默认为 \texttt{void}。如果尝试对 \texttt{void} 对象调用方法,会触发运行时错误 ("Dispatch to void")。这是与 Java (NullPointerException) 类似但处理方式不同的机制。
	
	\textbf{解决方案:} \\
	采用了防御性编程策略,使用 \texttt{isvoid} 谓词进行严格检查。例如,在 \texttt{BST} 类中,必须先检查 \texttt{root} 是否为空,才能调用其方法:
	
	\begin{lstlisting}
		insert(val : Int) : Object {
			if (isvoid root) then
			root <- (new Node).init(val)
			else
			root.insert(val)
			fi
		};
	\end{lstlisting}
	
	\section{测试与结果}
	
	\subsection{测试代码 (Main 类)}
	\texttt{Main} 类构建了一棵树,并以乱序 (50, 30, 20, 40, 70, 60, 80) 插入数据,以测试二叉树的自动排序特性。
	
	\begin{lstlisting}[caption={Main 类中的测试序列}]
		out_string("Inserting: 50, 30, 20, 40, 70, 60, 80\n");
		tree.insert(50); tree.insert(30); tree.insert(20);
		tree.insert(40); tree.insert(70); tree.insert(60); tree.insert(80);
		
		out_string("In-order Traversal: ");
		tree.print_in_order();
	\end{lstlisting}
	
	\subsection{运行结果}
	程序使用 \texttt{coolc} 编译并在 \texttt{spim} MIPS 模拟器上运行。输出结果显示,中序遍历成功输出了有序序列,验证了 BST 的性质。
	
	\begin{figure}[H]
		\centering
		\includegraphics[width=0.8\textwidth]{image.png} % 将文本输出替换为图片
		\caption{COOL 程序运行结果截图}
		\label{fig:terminal_output}
	\end{figure}
	
	\subsection{结果分析}
	\begin{enumerate}
		\item \textbf{排序正确性}: 尽管输入是乱序的,输出序列 \texttt{20 30 40 50 60 70 80} 是严格递增的。这证明了 \texttt{insert} 方法正确地维护了二叉搜索树的左小右大性质。
		\item \textbf{查找准确性}: 对存在节点 (40, 20) 的查找返回了 \texttt{true},对不存在节点 (90) 的查找返回了 \texttt{false},验证了递归查找逻辑的正确性。
	\end{enumerate}
	
	\section{结论}
	
	我在做这个作业的时候,首先要通过查找一些详细的资料去充分地认识到cool语言的语法特征。cool语言是为了体现“编译”特性而产生的语言,与以往学过的主流语言有很大的区别:
	\begin{enumerate}
		\item 类型系统采用严格的静态强类型且必须显式声明,不支持隐式转换;
		\item 继承机制为单继承且所有类隐式继承自Object类,不支持接口或抽象类;
		\item 空值处理使用独特的void关键字和isvoid运行时检查,而非null;
		\item 语法设计刻意简化,如使用if-then-else-fi和loop-pool结构,且赋值操作使用<-符号;
		\item 语言特性缺失较多,如不提供数组、异常处理、运算符重载等常见特性;
		\item 执行环境依赖编译为MIPS汇编并在SPIM模拟器中运行,而非直接生成机器码或字节码。
	\end{enumerate}
	
	而对于本次作业,实现二叉搜索树这样的递归数据结构,凸显了在无显式空指针语言中进行防御性编程(检查 \texttt{void})的重要性。本项目成功满足了所有功能要求,包括挑战项中的二叉搜索树实现。
\end{document}